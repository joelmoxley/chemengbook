\begin{mitframe}{001-1} % 20160703MAC

    \mdfsubtitle{Physics (Box 1-1)} 
{	\centering
\begin{listone}
    
    \item Descriptions of Matter, Energy, and Momenta
    
    \item Macro- and Micro- Views
    
\end{listone}

        
Types of work interactions
                
$\delta\work = \force \cdot \mathrm{d}\gendisp$
        
\bigskip
        
        %not sure if the delta symbol the boundary layer thickness ('\blthick')? underlined x the extensive vector ('\exvecone')?
        
        % (from joel) this is just a differential (hal double check me)
        % 20160703MAC: this is just a differential (Tester, p.22, Eq. 3-2)
        
        % i think the exvecone is the right usage (hal double check me)
        % 20160703MAC: underlined x is a 'general displacement vector' (Tester, p.xvi & p.22) <resolved>
                
\begin{tabular}{ | c | c |  }
        
	\hline
            
	Type of work & $\force \cdot \mathrm{d}\gendisp$ \\
    % 20160703MAC: underlined x should be a 'general displacement vector' (Tester, p.xvi & p.22) <resolved>
            
	\hline
            
	Pressure-Volume & $-\p \cdot \mathrm{d}\uline{\vol}$ \\
            
	\hline
            
	Frictional & $\fricforce \cdot \mathrm{d}\gendisp$ \\
    % 20160703MAC: underlined x should be a 'general displacement vector' (Tester, p.xvi & p.22) <resolved>
            
	\hline
            
	Surface Deformation & $\surftens \cdot \mathrm{d}\totalsa$ \\
            
	\hline  
    %not sure if the sigma symbol is the boundary quantity. not sure if 'a' is activity
    % 20160630MAC: sigma here is surface tension [N/m] and underlined 'a' is total surface area [m^2] (Tester, p.xv & xvii). <resolved>
            
	Electrical Charge Transport & $\elecpot \cdot \mathrm{d}\eleccharge$ \\
            
	\hline  
    % 20160706MAC: 'epsilon' here is not '\dielec', it is defined as "electric potential" or "electromotive force" <resolved>
    
    %not sure what 'q' symbolizes (is it heat flux?).
    % 20160630MAC: 'q' here refers to electric charge (Tester, p.31 & p.xvi) <resolved>
            
	Electric Polarization & $\elecfield \cdot \mathrm{d}\uline{\elecdisplace}$ \\
            
	\hline
            
	Magnetic Polarization & $\magfield \cdot \mathrm{d}\magind$ \\ \hline
            
	Stress-Strain & $\origvol(\sfrac{\force_{\disp}}{\totalsa})\mathrm{d}\linstrain = \origvol(\sfrac{\force_{\disp}}{\totalsa})\sfrac{\mathrm{d}\disp}{\disp_{\init}} $ \\ 
% * <floresbenjie@gmail.com> 2016-07-11T10:41:57.646Z:
%
% > \force_{\disp}
%
% ^.    
    \hline  
    %not sure what '\uline{\vol}_{0}' is correct. This symbol is not indexed.
    % 20160703MAC: \uline{\vol}_{o}' means 'original volume' (Tester, p.31)
    % 20160704MAC: i think this should be indexed as one (i.e. '\origvol') just like '\fricforce' or friction force instead of {\vol}_{0} <resolved>
    
    % 20160703MAC: 'omega' is not '\mpc', it means 'one-dimensional strain' or linear strain (Tester, p.xvii & p.31) <resolved>
    
    % 20160703MAC: replace '\activity'. underlined 'a' is total surface area [m^2] (Tester, p.xvii & p.31). this is similar with 'S_A' in the index <resolved>
            
\end{tabular}

}
\end{mitframe}