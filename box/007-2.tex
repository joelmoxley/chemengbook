\begin{mitframe}{007-2} % 20160707MAC

    \mdfsubtitle{Potential Fundamentals (Box 7-2)}
    
    % 20160707MAC: Tester, p.411 for Coulombic and Dipole-Dipole attraction
    
    \begin{listone}
    
    	\item $\force = -\dfrac{\mathrm{d}\forcepot}{\mathrm{d}\pos}$      
      	% 20160707MAC: F = -du/dr, sign should be negative. <resolved>
     	% 20160707MAC: 'u' here should be 'phi' or the intermolecular potential. <resolved>

      	\item Coulombic interaction: $\forcepot \sim \dfrac{\charge{i} \charge{j}}{\pos}$
	
     	\item Dipole-Dipole: $\forcepot_{d-d} \sim \dfrac{-\partpoten_{dipole}}{\pos^6\boltz\Temp}$      
      	% 20160707MAC: 'k' is not '\thermcond', it is the Boltzmann constant <resolved>
      
      	% 20160707MAC: all 'epsilon' are not '\dielec nor 'energy well depth' but just coefficients that depend on the identity of the molecules (e.g. "coefficient of dipole-dipole", "coefficient of dispersion-attraction", "coefficient of repulsion"). Some references describe as just constants.
      	% epsilon should be defined as particle potential energy in Joules [J] <>unresolved<>
        % 20161019BFF: added a new quantity in quantity_commands.tex, \partpoten for 'particle potential energy' with units [J]

      	\item Dispersion-Attraction: $\forcepot_{att} \sim \dfrac{-\partpoten_{att}}{\pos^6}$

     	\item Repulsion: $\forcepot_{rep} \sim \dfrac{-\partpoten_{rep}}{\pos^{12}}$
      
      	% 20160707MAC: exponent is usually 12, can be written more generally as '\pos^{n}' where n = 9 to 100 (Tester, p.413, Eq.10-88)    
    \end{listone}
 
\end{mitframe}